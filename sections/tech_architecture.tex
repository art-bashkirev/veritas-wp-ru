\subsection{Обзор проектирования системы}

\textbf{Подход к микросервисам:}
\noindent

Veritas использует архитектуру микросервисов, в которой отдельные службы выполняют отдельные функции:
\noindent
\begin{itemize}
    \item \textbf{Служба судьи:} Получает материалы, проверяет их формат, выбирает подходящий контейнер и инициирует распределенное выполнение.
    \item \textbf{Служба прогонщиков:} Выполняет тестовые примеры в изолированных контейнерах и генерирует вердикты.
    \item \textbf{Агрегатор вердиктов:} Собирает вердикты от прогонщиков, выполняет дополнительный анализ и сохраняет окончательные результаты.
    \item \textbf{Сервис пользовательского интерфейса:} Взаимодействует с пользователями, обрабатывает аутентификацию, отображает статус отправки и предоставляет обратную связь.
    \item \textbf{Сервис управления данными:} Управляет доступом и взаимодействием с распределенными базами данных (Redis, PostgreSQL, MongoDB).
\end{itemize}

\noindent
Каждый сервис работает независимо, взаимодействуя с другими через определенные API. Это позволяет:

\begin{itemize}
    \item \textbf{Масштабируемость:} Отдельные сервисы можно масштабировать горизонтально, добавляя новые экземпляры в зависимости от спроса.
    \item \textbf{Управляемость:} Маленькие, сфокусированные сервисы легче разрабатывать, тестировать и обновлять.
    \item \textbf{Устойчивость к сбоям:} Сбои в работе сервисов изолированы, что минимизирует влияние на всю систему.
\end{itemize}

**Механизмы связи:**.

Veritas использует различные механизмы связи в зависимости от контекста:

* **RESTful API:** Используется для обмена структурированными данными между службами, например для получения представлений или хранения вердиктов.
* **Websockets:** Обеспечивают двунаправленную связь между клиентами и сервером в реальном времени, предоставляя оперативные обновления статуса подачи.
**Redis Pub/Sub:** Обеспечивает асинхронную передачу сообщений между сервисами. Прогонщик публикует вердикт по завершении работы, а агрегатор вердиктов подписывается на его получение.
* ** Очереди сообщений:** Для сценариев, требующих буферизации сообщений, будут реализованы ю RabbitMQ.

Эти механизмы обеспечивают эффективную и масштабируемую связь в распределенной системе.

**Дополнительные соображения:**

* **Обнаружение сервисов:** Такие механизмы, как Consul или Kubernetes, могут быть использованы для динамического определения местоположения сервисов.
* **Управление конфигурацией:** Такие инструменты, как Puppet или Chef, позволяют управлять конфигурацией распределенных сервисов.
* **Логгинг и мониторинг:** Распределенные сервисы протоколирования, такие как ELK Stack, обеспечивают централизованное понимание поведения системы и потенциальных проблем.

**Преимущества проектирования распределенных систем:**

* **Масштабируемость:** Эффективно справляется с высокой пользовательской нагрузкой и скачками трафика.
* **Надежность:** Избыточность и изоляция сводят к минимуму время простоя и сбои в обслуживании.
* **Производительность:** Быстрая и эффективная связь обеспечивает низкую задержку и быстрое время отклика.
* **Устойчивость:** Модульные сервисы упрощают разработку, тестирование и обновление.

\subsection{Контайнеризация}

\subsection{Управление распределенными данными}

\subsection{Выделенный реестр контайнеров}

\subsection{Соединение в реальном времени}

\subsection{Минимизация и оптимизация}

\subsection{Проверка с помощью искуственного интеллекта}

