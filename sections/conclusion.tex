\subsection{Вывод}
Система в своей начальной форме в виде версии 0.0.1 является примером того, как можно исследовать новые концепции. 
Хотя она еще не охватывает весь спектр предполагаемых функций, эта начальная итерация служит пилотным исследованием распределенных систем судейства кода.
% В контексте академического проекта Veritas способствовал получению обширного опыта, углубляясь в тонкости распределенных архитектур, контейнеризации и потенциала проверки на основе искусственного интеллекта.
В контексте академического проекта Veritas способствовал получению обширного опыта, углубляясь в тонкости распределенных архитектур, контайнеризации и их реализации. \newline

\noindent
Хотя на начальном этапе реализации Veritas не удалось получить исчерпывающие количественные данные, проект заложил прочный фундамент для дальнейшего развития и оценки.
Несмотря на эти трудности, были получены ценные сведения, которые заложили основу для создания надежной и эффективной платформы для оценки кода.

\noindent
\textbf{Полученный опыт в жизненном цикле проекта:}
\begin{itemize}
    \item \textbf{Важность детального планирования:} Опыт показал необходимость тщательного планирования сценариев нагрузочного тестирования и контрольных показателей производительности с самого начала.
    \item \textbf{Устранение узких мест в инфраструктуре:} Начальное выявление потенциальных узких мест в производительности во время разработки позволило выбрать архитектурный дизайн для обеспечения масштабируемости и быстродействия.
    \item \textbf{Важность тестирования в реальных условиях:} Проект подчеркнул важность включения тестирования в реальных условиях, когда это возможно.
\end{itemize}