% \subsection{Ожидаемые преимущества}

% \subsection{План дальнейшего развития}

Метрики скорости агрегации, задержки и времени отклика имеют первостепенное значение для судьи по коду, существенно влияя на удобство использования и справедливость системы.

\textbf{Задачи оценки:}
\noindent
\begin{itemize}
    \item Количественно оценить способность системы справляться с различной нагрузкой при сохранении приемлемого времени отклика и задержки.
    \item Выявить и устранить потенциальные узкие места в производительности для проактивной оптимизации.
\end{itemize}

\textbf{Целевые функции:}
\noindent
\begin{itemize}
    \item \textbf{Скорость агрегации вердиктов:} измерение времени, необходимого агрегатору вердиктов для эффективного сбора и консолидации вердиктов от отдельных исполнителей. Эта метрика напрямую влияет на общее время ожидания пользователями результатов. (Возможно только для разрабатываемой платформы)
    \item \textbf{Задержка обратной связи в реальном времени:} оценка задержки между завершением отправки и предоставлением обратной связи пользователям (если применимо). Минимизация этого времени имеет решающее значение для подобных систем.
    \item \textbf{Время отклика системы:} оценка общей скорости реакции системы на действия пользователей. Сюда входят такие действия, как загрузка заявок, просмотр результатов и взаимодействие с платформой. Быстрое время отклика способствует плавному и бесперебойному взаимодействию с пользователем.
\end{itemize}

\textbf{Количественные показатели для разрабатываемой системы:}
\noindent
\begin{itemize}
\item \textbf{Скорость агрегирования вердиктов:}
    \begin{itemize}
    \item Среднее и максимальное время агрегации при различных моделируемых пользовательских нагрузках.
    \item Анализ распределения времени агрегации для выявления потенциальных выбросов или изменений производительности.
    \end{itemize}
\item \textbf{Задержка обратной связи в реальном времени:}
    \begin{itemize}
    \item Средняя и максимальная задержка между завершением отправки и получения вердиктов.
    \item Измерение задержки в различных сетевых условиях и географических регионах.
    \end{itemize}
\item \textbf{Время отклика системы:}
    \begin{itemize}
    \item Среднее и максимальное время отклика для типичных действий пользователя при разных уровнях нагрузки.
    \item Разбивка времени отклика по конкретным компонентам системы (например, интерфейсная часть, серверная часть, база данных) для целевой оптимизации.
    \end{itemize}
\end{itemize}

\textbf{Методика нагрузочного тестирования:}

Чтобы создать реалистичные сценарии и точно оценить производительность в различных условиях, планируется использовать комбинацию подходов:
\begin{itemize}
    \item \textbf{Инструменты тестирования контролируемой нагрузки:} Использование инструментов Apache JMeter, Locust и K6 для имитации одновременных пользователей, отправляющих решения и получающих вердикты.
    \item \textbf{Контролируемые эксперименты:} Проведение экспериментов, стратегически изменяя количество одновременных пользователей, частоту отправки и сложность компиляции и прогона, чтобы проанализировать их влияние на показатели производительности.
    \item \textbf{Реальное использование (если возможно):} Привлечение реальных пользователей к контролируемому бета-тестированию и сбор отзывов о воспринимаемой реакции в реальных условиях.
\end{itemize}